% !TeX root = doc.tex
\documentclass{article}
\usepackage[utf8]{inputenc}
\usepackage[T1]{fontenc}
\usepackage{amsmath}
\usepackage{minted}[cache=False]

\setminted{linenos=true}
\title{Tytuł}
\author{Patryk Wałach}
\date{January 2022}

\begin{document}

\maketitle
\tableofcontents

\section{[Wstęp]}

\section{Ogólne wprowadzenie}
\subsection{Analiza leksykalna}
Analiza leksykalna w informatyce jest to proces rozbijania program źródłowych na jednostki logiczne (zwane leksemami) złożone z jednego lub więcej znaków, które łącznie mają jakieś znaczenie\cite{Hopcroft__Motwani__Ullman__2005}. Przykładami leksemów mogą być słowa kluczowe (np. while), identyfikator lub liczba składająca się z cyfr. Rozdzielaniem programu źródłowego, na leksemy, zajmuje się lekser.

Token jest strukturą reprezentującą leksem i wprost go kategoryzującą\cite{ Aho__Sethi__Ullman__1985}, co ułatwia późniejszą pracę parserowi. Tokeny kategoryzuje się na komputerowy odpowiednik tego, co lingwiści określiliby mianem części mowy. Biorąc jako przykład poniższy kod w języku C:
\begin{minted}{c}
  x = a + b * 2;
\end{minted}
analiza leksykalna, zwraca tokeny:
\begin{minted}{python}
  [(identifier, x), (operator, =), (identifier, a), (operator, +), (identifier, b), (operator, *), (literal, 2), (separator, ;)]
\end{minted}
Dwoma ważnymi przypadkami są znaki białe i komentarze. One również muszą być uwzględnione w gramatyce i przeanalizowane przez lexer, lecz mogą być odrzucone (nie produkować żadnych tokenów) i traktowane jako spełniające mało znaczące zadanie, rozdzielania dwóch tokenów (np. w \mintinline{uwu.py -x}{if x} zamiast \mintinline{uwu.py -x}{ifx}).

\subsection{Parser}
Analizator składniowy, parser – program dokonujący analizy składniowej danych wejściowych w celu określenia ich struktury gramatycznej w związku z określoną gramatyką formalną. Analizator składniowy umożliwia przetworzenie tekstu czytelnego dla człowieka w strukturę danych przydatną dla oprogramowania komputera. Wynikiem analizy składni, dokonywanej przez parser, najczęściej jest drzewo składniowe nazywane czasami drzewem wyprowadzenia\cite{Aho__Sethi__Ullman__1985}.


\newpage
Zadanie parsera sprowadza się do sprawdzenia czy i jak dane wejściowe mogą zostać wyprowadzone z symbolu startowego. To zadanie można zrealizować na dwa sposoby:


\begin{itemize}
  \item Analiza zstępująca (ang. top-down parsing) to strategia znajdowania powiązań między danymi przez stawianie hipotez dotyczących drzewa rozbioru składniowego i sprawdzanie, czy zależności między danymi są zgodne z tymi hipotezami.
  \item Analiza wstępująca (ang. bottom-up parsing) – ogólna metoda analizy składniowej, w której zaczyna się od słowa wejściowego i próbuje się zredukować je do symbolu startowego. Drzewo wyprowadzenia jest konstruowane od liści do korzenia (stąd nazwa). W każdym momencie w trakcie tego procesu mamy formę zdaniową, która zawiera segment, powstały w ostatnim kroku wyprowadzenia. Segment ten nazywany uchwytem (ang. handle) jest prawą stroną produkcji i powinien zostać w tym kroku zredukowany do jej lewej strony, w wyniku czego powstanie poprzednia forma zdaniowa z wyprowadzenia. Główna trudność w analizie wstępującej polega właśnie na odpowiednim znajdywaniu uchwytów.
        Analiza wstępująca może przebiegać w określonym kierunku (np. od lewej do prawej), lub w sposób bezkierunkowy, wtedy analizowane jest całe słowo naraz. Jednym z bardziej znanych przedstawicieli metody bezkierunkowej jest algorytm CYK. Do metod kierunkowych zalicza się między innymi parsery shift-reduce czyli LR, LALR, SLR, BC, pierwszeństwa.
\end{itemize}



\subsection{System typów}
System typów jest to system klasyfikacji wyrażeń w zależności od rodzajów wartości, jakie one generują\cite{Pierce__Benjamin__C__2002}. Każdej obliczonej wartości przypisywany jest pewien typ, który jednoznacznie definiuje, jakie operacje można na niej wykonać. Śledząc przepływ wartości, system typów stara się udowodnić, że w programie występuje poprawne typowanie, tzn. nie dochodzi do sytuacji, w której na wartości określonego typu próbujemy wykonać niedozwoloną operację.
\subsection{System typów ML} System typów ML jest to silny system typów stosowany w językach rodziny ML (Ocaml, Standard ML) oparty na inferencji.

Podstawowy system typów jest następujący: istnieją typy proste, takie jak string, int, bool, unit (typ pusty) itd. Z dowolnych typów można też generować typy złożone – przez krotki (typ1 * typ2, typ1 * typ2 * typ3 itd.), konstruktory typów (typ list, typ tree itd.) i funkcje (typ1 → typ2).

System próbuje nadać typy każdemu wyrażeniu języka, i nie licząc kilku rzadkich przypadków, udaje mu się to całkiem dobrze.

Generalnie system taki wyklucza polimorfizm (nie licząc typów polimorficznych), jednak w Standard ML stworzono specjalne reguły umożliwiające polimorfizm dla wyrażeń arytmetycznych.

System typów ML jest interesujący z teoretycznego punktu widzenia – wiele problemów ma bardzo wysoką złożoność, jednak w praktyce inferencja zachodzi bardzo szybko – typy, które są rzeczywiście używane, są zwykle bardzo proste – rzadko używa się funkcji rzędów wyższych niż trzeci-czwarty, oraz liczby argumentów większej niż kilkanaście.

W rzeczywistych implementacjach dochodzą do tego bardziej złożone problemy typizacji obiektów, modułów itd.

\subsection{Rekord z wariantami} Rekord z wariantami jest to rodzaj rekordu, posiadającego tę właściwość, że zbiór rekordów posiada wspólny typ, lecz różną postać, określoną aktualną wartością specjalnego pola znacznikowego.

Przykład - Zakładając, że checmy stworzyć drzewo binarne intów. W języku ML, zrobilibyśmy to tworząc nowy typ danych w ten sposób:
\begin{minted}{ocaml}
datatype tree = Leaf
  | Node of (int * tree * tree)
\end{minted}
\mintinline{uwu.py -x}{Leaf} i \mintinline{uwu.py -x}{Node} są konstruktorami, które pozwalają nam na stworzenie konretnego drzewa, np.
\begin{minted}{ocaml}
  Node(5, Node(1, Leaf, Leaf), Node(3, Leaf, Node(4, Leaf, Leaf)))
\end{minted}
\section{Inferencja typów w teori}
\subsection{Problem inferencji typów}
Język ML przyjmuje wiele form, najpopularniejszymi wariantami są Standard ML (SML), OCaml,
i F\#. Na potrzeby będziemy wzorować się na \cite{Damas__Milner__1982}, i posługiwać się ML-the-calculus, drastycznie uproszczoną wersją języka co pozwoli na dojście do sedna w problemie rekonstrukcji typów.

Termy w języku ML-the-calculus są następujące:
\begin{equation}
  \begin{split}
    e\ ::&=x\\
    &|\ \ \ c\qquad(\text{stałe})\\
    &|\ \ \ \lambda x.e\\
    &|\ \ \ e\ e\\
    &|\ \ \ \text{let }x=e\text{ in }e
  \end{split}
\end{equation}
Typy które będziemy przypisywali do termów są następujące:
\begin{equation}
  \begin{split}
    \tau\ ::&=\alpha\qquad(\text{typ zmienny})\\
    &|\ \ \ B\qquad(\text{typ podstawowy})\\
    &|\ \ \ \tau\rightarrow\tau
  \end{split}
\end{equation}
\subsection{Udowadnianie typów dla ML-the-calculus}

\begin{equation}
  \begin{split}
    \frac{}{\Gamma\vdash c:B} \\
    \\
    \frac{\Gamma,x:\tau'\vdash e:\tau}{\Gamma\vdash\lambda x.e:\tau'\rightarrow\tau} \\
    \\
    \frac{\Gamma\vdash e_1:\tau_2\rightarrow\tau\qquad\Gamma\vdash e_2:\tau_2}{\Gamma\vdash e_1\ e_2:\tau} \\
    \\
    \frac{\Gamma(x)=\bigwedge\alpha_1,...,\alpha_n.\tau'\qquad\tau=[\beta_i/\alpha_i]\tau'}{\Gamma\vdash x:\tau}(\beta_1\text{ fresh}) \\
    \\
    \frac{\Gamma\vdash e_1:\tau'\qquad\Gamma,x:(\bigwedge\alpha_1,...,\alpha_n.\tau')\vdash e_2:\tau}{\Gamma\vdash \text{let }x=e_1\text{ in }e_2:\tau:B}(\{\alpha_1,...,\alpha_n\}=ftv(\tau')\backslash ftv(\Gamma)) \\
  \end{split}
\end{equation}
Powyższe dowody nie mogą jednak być w prosty sposób przedstawione jako algorytm. Fakt, że typ $\tau'$ jest dowolnie wybierany w dowodzie dla $\lambda$ prowadzi do nieskończonej ilość dowodów, nawet dla prostego wyrażenia $\lambda x.x$.
\subsection{Infernencja typów bazująca na ograniczeniach}
Algorytmy infernencji typów bazujące na ograniczeniach generują dużą liczbę zmiennych typów i zbiór ograniczeń dla tych zmiennych. W drugim kroku algorytmu dla każdej zmiennej, szuka typu, który spełnia wszystkie ograniczenia.
\subsection{System typów Hindley–Milner}
Algorytm inferencji typów przedstwiony w \cite{Milner__1978}, oraz innych wcześniejszych publikacjach, opiera się na generowaniu ograniczeń i rozwiązywaniu ich w trakcie wykonywania prostego rekursywnego przejścia przez termy.

Warto zauważyć, że w ten sposób: dostajemy, prostą strukturalnie rekursywną definicjię rekonstrukcji typów, ale tracimy modularność algorytmu. Rozszerzenie takiego algorytmu o dodatkowe funkcjie jest o wiele trudniejsze, niż algorytmu, który oddziela kroki generowania ograniczeń i ich rozwiązywania.
\newpage
\subsection{Algorytm J}
Algorytm W nie ma żadnych side-effectów, i świetnie zajmuje się aplikowanie i komponowaniem substytucji w odpowiedniej kolejności. Częste apikacje substytucji na wyrażeniach mogą znacznie zmniejszyć wydajność algorytmu rekonstruującegom dlatego też, Milner zaprezentował bardziej efektywną impereatywną wariacją W nazwaną algorytmem J w \cite{Milner__1978}.

Algorytm J jest funkcją, która dla dokonanych do tej pory substytucji $S$, środowiska $\Gamma$ - zbioru przechowywującego pary identyfikator wraz z typem, i wyrażnia $e$, zwraca kolejne substytucje $S$, oraz typ wyrażenia $\tau$.
\begin{equation}
  \begin{split}
    J :& S\times\Gamma\times e \rightarrow S\times\tau \\
    \\
    J(S,\Gamma,x) =& (S,[\beta_i/\alpha_i]\tau') \\
    &\text{where } \Gamma(x) = \bigwedge\alpha_i,...,\alpha_n.\tau' \\
    &\text{and } \beta_i \text{ are fresh} \\
    \\
    J(S,\Gamma,e_1\ e_2) =& (V,\beta) \\
    &\text{where } \Gamma(S_1,\tau_1) = J(S,\Gamma,e_1) \\
    &\text{and } (S_2,\tau_2)=J(S_1,\Gamma,e_2) \\
    &\text{and } V = unify'(\tau_1,\tau_2\rightarrow\beta,S_2) \\
    &\text{and } \beta \text{ is fresh} \\
    \\
    J(S,\Gamma,\lambda x.e) =& (S_1,\beta\rightarrow\tau) \\
    &\text{where } (S_1,\tau)=J(S,(\Gamma,x:\beta),e) \\
    &\text{and } \beta \text{ is fresh} \\
    \\
    J(S,\Gamma,\text{let }x=e_1\text{ in }e_2) =& (S_2,\tau_w) \\
    &\text{where } (S_1,\tau_1)=J(S,\Gamma,e_1) \\
    &\text{and } (S_2,\tau_2)=J(S_1,(\Gamma,x:(\bigwedge\alpha_1,...,\alpha_n.\tau_1)),e_2) \\
    &\text{and } \{\alpha_1,...,\alpha_n\}=ftv(S_2\tau_1)\backslash ftv(S_2\Gamma) \\
  \end{split}
\end{equation}
Funkcja pomocnicza $unify'(\tau,\tau',S)$ rozszerza zbiór substytucji S o substytucje wynikające z unifikacji $\tau$ z $\tau'$ pod kontekstem $S$.


\section{Rescript/ReasonML jako języki realizujące podobne zadania}
\newpage
\section{Założenia i priorytety opracowanej aplikacji}
Tworząc aplikację, chciałem, by język posiadał podstawowe typy danych (liczby, stringi, wartość logiczna), kilka typów generycznych (funkcje, tablice), typ `Option', oraz możliwość tworzenia własnych typów.
Dodatkowo nie powinno być potrzeby podawania typów zmiennych w większości przypadków, kompilator sam powinien wykrywać typy zmiennych na podstawie ich użycia.


Język poza zmiennymi, potrzebuje możliwości wykonywania operacji na danych, dlatego ważne było dla mnie, by zaimplementować operatory binarane, oraz unarne. Operatory te miały też spełniać ważną rolą w trakcie inferencji typów. W języku javascript operator `+' może być wykorzystywany do dodawania liczb jak i konkatenacji stringów, ważne więc było by stworzyć dwa oddzielne operatory.

Prymitywne typy danych:
\begin{itemize}
  \item string
        \inputminted{uwu.py -x}{examples/string.uwu}
        % Do konkatenacji stringów służy operator \mintinline{uwu.py -x}{++}
  \item Wartość logiczna ma typ \mintinline{uwu.py -x}{Bool} i wartość \mintinline{uwu.py -x}{True} lub \mintinline{uwu.py -x}{False}.
        Powiązane operacje:
        \begin{itemize}
          \item \mintinline{uwu.py -x}{<>}, równość pomiędzy dwiema liczbami
          \item \mintinline{uwu.py -x}{>}, \mintinline{uwu.py -x}{<}
          \item \mintinline{uwu.py -x}{!=} równość
        \end{itemize}
  \item liczby
        % Powiązane operacje: \mintinline{uwu.py -x}{+}, \mintinline{uwu.py -x}{-}, \mintinline{uwu.py -x}{*}, \mintinline{uwu.py -x}{/}, \mintinline{uwu.py -x}{*}, \mintinline{uwu.py -x}{\%}, \mintinline{uwu.py -x}{//}
\end{itemize}

% typing.Literal["|", "!=", "=="]

\newpage
Chciałem również by funkcje wieloargumentowe kompilowane były jako funkcje jednoargumentowe zwracające kolejne funkcje, co pozwala na wywoływanie funkcji bez wszystkich argumentów w celu zwrócenia funkcji przyjmującej resztę argumentów tzw. currying.

% \inputminted[firstline=15]{uwu.py -x}{examples/currying.uwu}
\inputminted[firstline=15]{javascript}{examples/currying.uwu.js}


Jednym z ważniejszych elemenów każdego języka jest możliwość wykonywania różnego zbioru instrukcji, warunkowo. W tym celu planowałem zaimplementowanie instrukcji `if', oraz `case'. Instrukcja `case' wykonywać ma dopasowanie do wzorca (tzw. pattern-matching), wykonywać, odpowiedni zbiór instrukcji zależnie od wprowadzonych danych. Kompilator, powinien ostrzegać, jeżeli ścieżka dla jednego z typów danych nie została zaimplementowana.

\begin{itemize}
  \newpage
  \item Przykład - funkcja łącząca dwie posortowane tablice
        \inputminted[firstline=25]{uwu.py -x}{examples/merge.uwu}
        \inputminted[firstline=20]{javascript}{examples/merge.uwu.js}
\end{itemize}

\newpage
Kolejnym dość ważnym elementem języka jest brak wyrażenia `return', które jest wykorzystywane do zwrócenia wartości z funkcji. Zamiast tego każdy bloku instrukcji powinien zwracać ostatnie wyrażenie. Pozwoli to na łatwiejsze inicjowanie zmiennych, w przypadku gdy inicializacja wymaga więcej niż jedenej linii kodu.



\begin{itemize}
  \item Przykład
        % \inputminted[firstline=15]{uwu.py -x}{examples/return.uwu}
        \inputminted[firstline=15]{javascript}{examples/return.uwu.js}
\end{itemize}
\subsection{Opis formaly składni języka}
%  notacja wirta
\section{Narzędzia}
\subsection{Język python}
Do implementacji programu postanowiliśmy wykorzystać język python w wersji 3.10, ze względu na jego dynamiczność.
W tej wersji języka pojawił się również pattern-matching, który znacząco ułatwia pracę z ast.
\subsection{Parsowanie i tokenizowanie przy użyciu biblioteki sly}
Biblioteka sly, jest pythonową implementacją narzędzi lex i yacc, wykorzystywanych do tworzenia parserów i kompilatorów.
\subsection{Środowisko nodejs do uruchomienia skompilowanego kodu}
\section{Implementacja}
\subsection{lexer}
\subsection{parser}
% \inputminted[firstline=111]{python}{parser.py}
\subsection{inferencja typów}
Zmienny typ reprezentuję jako unikalną liczbę
\inputminted[firstline=28, lastline=34]{python}{algorithm_j.py}
Substytucje to pary zmiennych i typów
\inputminted[firstline=25, lastline=25]{python}{algorithm_j.py}
Każda zmienna w programie reprezentowana jest przez Scheme zawierający listę generycznych typów
\inputminted[firstline=26, lastline=26]{python}{algorithm_j.py}
Context to mapa identyfikatorów i schem
\inputminted[firstline=12, lastline=22]{python}{algorithm_j.py}
Tworzymy funkcję, aplikującą substytucje na typie
\inputminted[firstline=37, lastline=46]{python}{algorithm_j.py}
Tworzymy funkcję, unifikującą dwa typy
\inputminted[firstline=95, lastline=112]{python}{algorithm_j.py}
wraz z funkcją aplikującą substytucje na zmiennym typie
\inputminted[firstline=115, lastline=124]{python}{algorithm_j.py}
Inferencja termóm dla literałów jest trywialna
\inputminted[firstline=164, lastline=171]{python}{algorithm_j.py}
W przypadku identyfikatorów, musimy zastąpić ich generyczne typy nowymi zmiennymi typami
\inputminted[firstline=172, lastline=173]{python}{algorithm_j.py}
W przypadku bloku wyrażeń inferujemy typ każdego z nich i zwracamy ostatni z nich
\inputminted[firstline=185, lastline=192]{python}{algorithm_j.py}
Binarny pperator konkatenacji list, sprawdzamy czy obie listy są tego samego typu
\inputminted[firstline=257, lastline=263]{python}{algorithm_j.py}
Wywołując funkcję aplikujemy argumenty na nowy zmiennym typie po czym unifikujemy rezultat z typem wywoływanego wyrażenia
\inputminted[firstline=323, lastline=335]{python}{algorithm_j.py}

\subsection{kompilacja}
\section{Opis działania}
\subsection{Co działa}
\subsection{Uwagi co do obsługi błędów}
\section{[Podsumowanie]}
\section{[spisy -- rysunków, tabel, listingów ipt.]}

\bibliographystyle{plainnat}
\bibliography{doc}


\end{document}


