% !TeX root = doc.tex
\documentclass{article}
\usepackage[utf8]{inputenc}
\usepackage[T1]{fontenc}
\usepackage{amsmath}
\usepackage{listings}
\usepackage{color} %use color
\definecolor{mygreen}{rgb}{0,0.6,0}
\definecolor{mygray}{rgb}{0.5,0.5,0.5}
\definecolor{mymauve}{rgb}{0.58,0,0.82}
 
%Customize a bit the look
\lstset{ %
backgroundcolor=\color{white}, % choose the background color; you must add \usepackage{color} or \usepackage{xcolor}
basicstyle=\footnotesize, % the size of the fonts that are used for the code
breakatwhitespace=false, % sets if automatic breaks should only happen at whitespace
breaklines=true, % sets automatic line breaking
captionpos=b, % sets the caption-position to bottom
commentstyle=\color{mygreen}, % comment style
extendedchars=true, % lets you use non-ASCII characters; for 8-bits encodings only, does not work with UTF-8
frame=single, % adds a frame around the code
keepspaces=true, % keeps spaces in text, useful for keeping indentation of code (possibly needs columns=flexible)
keywordstyle=\color{blue}, % keyword style
% language=Octave, % the language of the code
morekeywords={*,...}, % if you want to add more keywords to the set
numbers=left, % where to put the line-numbers; possible values are (none, left, right)
numbersep=5pt, % how far the line-numbers are from the code
numberstyle=\tiny\color{mygray}, % the style that is used for the line-numbers
rulecolor=\color{black}, % if not set, the frame-color may be changed on line-breaks within not-black text (e.g. comments (green here))
showspaces=false, % show spaces everywhere adding particular underscores; it overrides 'showstringspaces'
showstringspaces=false, % underline spaces within strings only
showtabs=false, % show tabs within strings adding particular underscores
stepnumber=1, % the step between two line-numbers. If it's 1, each line will be numbered
stringstyle=\color{mymauve}, % string literal style
tabsize=2, % sets default tabsize to 2 spaces,
title=\lstname % show the filename of files included with \lstinputlisting; also try caption instead of title
}
%END of listing package%



\lstset{
  language=Python
  }

\title{Tytuł}
\author{Patryk Wałach}
\date{January 2022}

\begin{document}

\maketitle
\tableofcontents

\section{[Wstęp]}

\section{Ogólne wprowadzenie}
\subsection{Analiza leksykalna}
Analiza leksykalna w informatyce jest to proces rozbijania program źródłowych na jednostki logiczne (zwane leksemami) złożone z jednego lub więcej znaków, które łącznie mają jakieś znaczenie\cite{Hopcroft__Motwani__Ullman__2005}. Przykładami leksemów mogą być słowa kluczowe (np. while), identyfikator lub liczba składająca się z cyfr. Rozdzielaniem programu źródłowego, na leksemy, zajmuje się lekser.

Token jest strukturą reprezentującą leksem i wprost go kategoryzującą\cite{ Aho__Sethi__Ullman__1985}, co ułatwia późniejszą pracę parserowi. Tokeny kategoryzuje się na komputerowy odpowiednik tego, co lingwiści określiliby mianem części mowy. Biorąc jako przykład poniższy kod w języku C:
\begin{lstlisting}[language=c]
  x = a + b * 2;
\end{lstlisting}
analiza leksykalna, zwraca tokeny:
\begin{lstlisting}
  [(identifier, x), (operator, =), (identifier, a), (operator, +), (identifier, b), (operator, *), (literal, 2), (separator, ;)]
\end{lstlisting}
Dwoma ważnymi przypadkami są znaki białe i komentarze. One również muszą być uwzględnione w gramatyce i przeanalizowane przez lexer, lecz mogą być odrzucone (nie produkować żadnych tokenów) i traktowane jako spełniające mało znaczące zadanie, rozdzielania dwóch tokenów (np. w \lstinline{if x} zamiast \lstinline{ifx}).

\subsection{Parser}
Analizator składniowy, parser – program dokonujący analizy składniowej danych wejściowych w celu określenia ich struktury gramatycznej w związku z określoną gramatyką formalną. Analizator składniowy umożliwia przetworzenie tekstu czytelnego dla człowieka w strukturę danych przydatną dla oprogramowania komputera. Wynikiem analizy składni, dokonywanej przez parser, najczęściej jest drzewo składniowe nazywane czasami drzewem wyprowadzenia\cite{Aho__Sethi__Ullman__1985}.



Zadanie parsera sprowadza się do sprawdzenia czy i jak dane wejściowe mogą zostać wyprowadzone z symbolu startowego. To zadanie można zrealizować na dwa sposoby:


\begin{itemize}
  \item Analiza zstępująca (ang. top-down parsing) to strategia znajdowania powiązań między danymi przez stawianie hipotez dotyczących drzewa rozbioru składniowego i sprawdzanie, czy zależności między danymi są zgodne z tymi hipotezami.
  \item Analiza wstępująca (ang. bottom-up parsing) – ogólna metoda analizy składniowej, w której zaczyna się od słowa wejściowego i próbuje się zredukować je do symbolu startowego. Drzewo wyprowadzenia jest konstruowane od liści do korzenia (stąd nazwa). W każdym momencie w trakcie tego procesu mamy formę zdaniową, która zawiera segment, powstały w ostatnim kroku wyprowadzenia. Segment ten nazywany uchwytem (ang. handle) jest prawą stroną produkcji i powinien zostać w tym kroku zredukowany do jej lewej strony, w wyniku czego powstanie poprzednia forma zdaniowa z wyprowadzenia. Główna trudność w analizie wstępującej polega właśnie na odpowiednim znajdywaniu uchwytów.
        Analiza wstępująca może przebiegać w określonym kierunku (np. od lewej do prawej), lub w sposób bezkierunkowy, wtedy analizowane jest całe słowo naraz. Jednym z bardziej znanych przedstawicieli metody bezkierunkowej jest algorytm CYK. Do metod kierunkowych zalicza się między innymi parsery shift-reduce czyli LR, LALR, SLR, BC, pierwszeństwa.
\end{itemize}



\subsection{System typów}
System typów jest to system klasyfikacji wyrażeń w zależności od rodzajów wartości, jakie one generują\cite{Pierce__Benjamin__C__2002}. Każdej obliczonej wartości przypisywany jest pewien typ, który jednoznacznie definiuje, jakie operacje można na niej wykonać. Śledząc przepływ wartości, system typów stara się udowodnić, że w programie występuje poprawne typowanie, tzn. nie dochodzi do sytuacji, w której na wartości określonego typu próbujemy wykonać niedozwoloną operację.
\subsection{System typów ML} System typów ML jest to silny system typów stosowany w językach rodziny ML (Ocaml, Standard ML) oparty na inferencji.

Podstawowy system typów jest następujący: istnieją typy proste, takie jak string, int, bool, unit (typ pusty) itd. Z dowolnych typów można też generować typy złożone – przez krotki (typ1 * typ2, typ1 * typ2 * typ3 itd.), konstruktory typów (typ list, typ tree itd.) i funkcje (typ1 → typ2).

System próbuje nadać typy każdemu wyrażeniu języka, i nie licząc kilku rzadkich przypadków, udaje mu się to całkiem dobrze.

Generalnie system taki wyklucza polimorfizm (nie licząc typów polimorficznych), jednak w Standard ML stworzono specjalne reguły umożliwiające polimorfizm dla wyrażeń arytmetycznych.

System typów ML jest interesujący z teoretycznego punktu widzenia – wiele problemów ma bardzo wysoką złożoność, jednak w praktyce inferencja zachodzi bardzo szybko – typy, które są rzeczywiście używane, są zwykle bardzo proste – rzadko używa się funkcji rzędów wyższych niż trzeci-czwarty, oraz liczby argumentów większej niż kilkanaście.

W rzeczywistych implementacjach dochodzą do tego bardziej złożone problemy typizacji obiektów, modułów itd.

\subsection{Rekord z wariantami} Rekord z wariantami jest to rodzaj rekordu, posiadającego tę właściwość, że zbiór rekordów posiada wspólny typ, lecz różną postać, określoną aktualną wartością specjalnego pola znacznikowego.

Przykład - Zakładając, że checmy stworzyć drzewo binarne intów. W języku ML, zrobilibyśmy to tworząc nowy typ danych w ten sposób:
\begin{lstlisting}[language=ML]
datatype tree = Leaf
  | Node of (int * tree * tree)
\end{lstlisting}
\lstinline{Leaf} i \lstinline{Node} są konstruktorami, które pozwalają nam na stworzenie konretnego drzewa, np.
\begin{lstlisting}[language=ML]
  Node(5, Node(1, Leaf, Leaf), Node(3, Leaf, Node(4, Leaf, Leaf)))
\end{lstlisting}
\section{Inferencja typów w teori}
\subsection{Problem inferencji typów}
Język ML przyjmuje wiele form, najpopularniejszymi wariantami są Standard ML (SML), OCaml,
i F\#. Na potrzeby będziemy wzorować się na \cite{Damas__Milner__1982}, i posługiwać się ML-the-calculus, drastycznie uproszczoną wersją języka co pozwoli na dojście do sedna w problemie rekonstrukcji typów.

Termy w języku ML-the-calculus są następujące:
\begin{equation}
  \begin{split}
    e\ ::&=x\\
    &|\ \ \ c\qquad(\text{stałe})\\
    &|\ \ \ \lambda x.e\\
    &|\ \ \ e\ e\\
    &|\ \ \ \text{let }x=e\text{ in }e
  \end{split}
\end{equation}
Typy które będziemy przypisywali do termów są następujące:
\begin{equation}
  \begin{split}
    \tau\ ::&=\alpha\qquad(\text{typ zmienny})\\
    &|\ \ \ B\qquad(\text{typ podstawowy})\\
    &|\ \ \ \tau\rightarrow\tau
  \end{split}
\end{equation}
\subsection{Udowadnianie typów dla ML-the-calculus}

\begin{equation}
  \begin{split}
    \frac{}{\Gamma\vdash c:B} \\
    \\
    \frac{\Gamma,x:\tau'\vdash e:\tau}{\Gamma\vdash\lambda x.e:\tau'\rightarrow\tau} \\
    \\
    \frac{\Gamma\vdash e_1:\tau_2\rightarrow\tau\qquad\Gamma\vdash e_2:\tau_2}{\Gamma\vdash e_1\ e_2:\tau} \\
    \\
    \frac{\Gamma(x)=\bigwedge\alpha_1,...,\alpha_n.\tau'\qquad\tau=[\beta_i/\alpha_i]\tau'}{\Gamma\vdash x:\tau}(\beta_1\text{ fresh}) \\
    \\
    \frac{\Gamma\vdash e_1:\tau'\qquad\Gamma,x:(\bigwedge\alpha_1,...,\alpha_n.\tau')\vdash e_2:\tau}{\Gamma\vdash \text{let }x=e_1\text{ in }e_2:\tau:B}(\{\alpha_1,...,\alpha_n\}=ftv(\tau')\backslash ftv(\Gamma)) \\
  \end{split}
\end{equation}
Powyższe dowody nie mogą jednak być w prosty sposób przedstawione jako algorytm. Fakt, że typ $\tau'$ jest dowolnie wybierany w dowodzie dla $\lambda$ prowadzi do nieskończonej ilość dowodów, nawet dla prostego wyrażenia $\lambda x.x$.
\subsection{Infernencja typów bazująca na ograniczeniach}
Algorytmy infernencji typów bazujące na ograniczeniach generują dużą liczbę zmiennych typów i zbiór ograniczeń dla tych zmiennych. W drugim kroku algorytmu dla każdej zmiennej, szuka typu, który spełnia wszystkie ograniczenia.
\subsection{System typów Hindley–Milner}
Algorytm inferencji typów przedstwiony w \cite{Milner__1978}, oraz innych wcześniejszych publikacjach, opiera się na generowaniu ograniczeń i rozwiązywaniu ich w trakcie wykonywania prostego rekursywnego przejścia przez termy.

Warto zauważyć, że w ten sposób: dostajemy, prostą strukturalnie rekursywną definicjię rekonstrukcji typów, ale tracimy modularność algorytmu. Rozszerzenie takiego algorytmu o dodatkowe funkcjie jest o wiele trudniejsze, niż algorytmu, który oddziela kroki generowania ograniczeń i ich rozwiązywania.

\subsection{Algorytm J}
Algorytm W nie ma żadnych efektów ubocznych, i świetnie zajmuje się aplikowanie i komponowaniem substytucji w odpowiedniej kolejności. Częste apikacje substytucji na wyrażeniach mogą znacznie zmniejszyć wydajność algorytmu rekonstruującegom dlatego też, Milner zaprezentował bardziej efektywną impereatywną wariacją W nazwaną algorytmem J w \cite{Milner__1978}.

Algorytm J jest funkcją, która dla dokonanych do tej pory substytucji $S$, środowiska $\Gamma$ - zbioru przechowywującego pary identyfikator wraz z typem, i wyrażnia $e$, zwraca kolejne substytucje $S$, oraz typ wyrażenia $\tau$.
\begin{equation}
  \begin{split}
    J :& S\times\Gamma\times e \rightarrow S\times\tau \\
    \\
    J(S,\Gamma,x) =& (S,[\beta_i/\alpha_i]\tau') \\
    &\text{where } \Gamma(x) = \bigwedge\alpha_i,...,\alpha_n.\tau' \\
    &\text{and } \beta_i \text{ are fresh} \\
    \\
    J(S,\Gamma,e_1\ e_2) =& (V,\beta) \\
    &\text{where } \Gamma(S_1,\tau_1) = J(S,\Gamma,e_1) \\
    &\text{and } (S_2,\tau_2)=J(S_1,\Gamma,e_2) \\
    &\text{and } V = unify'(\tau_1,\tau_2\rightarrow\beta,S_2) \\
    &\text{and } \beta \text{ is fresh} \\
    \\
    J(S,\Gamma,\lambda x.e) =& (S_1,\beta\rightarrow\tau) \\
    &\text{where } (S_1,\tau)=J(S,(\Gamma,x:\beta),e) \\
    &\text{and } \beta \text{ is fresh} \\
    \\
    J(S,\Gamma,\text{let }x=e_1\text{ in }e_2) =& (S_2,\tau_w) \\
    &\text{where } (S_1,\tau_1)=J(S,\Gamma,e_1) \\
    &\text{and } (S_2,\tau_2)=J(S_1,(\Gamma,x:(\bigwedge\alpha_1,...,\alpha_n.\tau_1)),e_2) \\
    &\text{and } \{\alpha_1,...,\alpha_n\}=ftv(S_2\tau_1)\backslash ftv(S_2\Gamma) \\
  \end{split}
\end{equation}
Funkcja pomocnicza $unify'(\tau,\tau',S)$ rozszerza zbiór substytucji S o substytucje wynikające z unifikacji $\tau$ z $\tau'$ pod kontekstem $S$.


\section{Rescript/ReasonML jako języki realizujące podobne zadania}

\section{Założenia i priorytety opracowanej aplikacji}
Tworząc aplikację, chciałem, by język posiadał podstawowe typy danych (liczby, stringi, wartość logiczna), kilka typów generycznych (funkcje, tablice), typ `Option', oraz możliwość tworzenia własnych typów.
Dodatkowo nie powinno być potrzeby podawania typów zmiennych w większości przypadków, kompilator sam powinien wykrywać typy zmiennych na podstawie ich użycia.


Język poza zmiennymi, potrzebuje możliwości wykonywania operacji na danych, dlatego ważne było dla mnie, by zaimplementować operatory binarane, oraz unarne. Operatory te miały też spełniać ważną rolą w trakcie inferencji typów. W języku javascript operator `+' może być wykorzystywany do dodawania liczb jak i konkatenacji stringów, ważne więc było by stworzyć dwa oddzielne operatory.

Prymitywne typy danych:
\begin{itemize}
  \item string
        \lstinputlisting{examples/string.uwu}
        Do konkatenacji stringów służy operator \lstinline!++!
  \item Wartość logiczna ma typ \lstinline!Bool! i wartość \lstinline!True! lub \lstinline!False!.
        Powiązane operacje:
        \begin{itemize}
          \item \lstinline!<>!, równość pomiędzy dwiema liczbami
          \item \lstinline!>!, \lstinline!<!
          \item \lstinline|!=| równość
        \end{itemize}
  \item liczby
        Powiązane operacje: \lstinline!+!, \lstinline!-!, \lstinline!*!, \lstinline!/!, \lstinline!*!, \lstinline!%!, \lstinline!//!
\end{itemize}

% typing.Literal["|", "!=", "=="]


Chciałem również by funkcje wieloargumentowe kompilowane były jako funkcje jednoargumentowe zwracające kolejne funkcje, co pozwala na wywoływanie funkcji bez wszystkich argumentów w celu zwrócenia funkcji przyjmującej resztę argumentów tzw. currying.

\lstinputlisting[firstline=15]{examples/currying.uwu}
\newpage
\lstinputlisting[firstline=15]{examples/currying.uwu.js}


Jednym z ważniejszych elemenów każdego języka jest możliwość wykonywania różnego zbioru instrukcji, warunkowo. W tym celu planowałem zaimplementowanie instrukcji `if', oraz `case'. Instrukcja `case' wykonywać ma dopasowanie do wzorca (tzw. pattern-matching), wykonywać, odpowiedni zbiór instrukcji zależnie od wprowadzonych danych. Kompilator, powinien ostrzegać, jeżeli ścieżka dla jednego z typów danych nie została zaimplementowana.

\begin{itemize}

  \item Przykład - funkcja łącząca dwie posortowane tablice
        \lstinputlisting[firstline=25]{examples/merge.uwu}\newpage
        \lstinputlisting[firstline=19]{examples/merge.uwu.js}
\end{itemize}


Kolejnym dość ważnym elementem języka jest brak wyrażenia `return', które jest wykorzystywane do zwrócenia wartości z funkcji. Zamiast tego każdy bloku instrukcji powinien zwracać ostatnie wyrażenie. Pozwoli to na łatwiejsze inicjowanie zmiennych, w przypadku gdy inicializacja wymaga więcej niż jedenej linii kodu.

\newpage
Przykład
\lstinputlisting[firstline=15]{examples/return.uwu}
\lstinputlisting[firstline=15]{examples/return.uwu.js}

\subsection{Opis formaly składni języka}
%  notacja wirta
\section{Narzędzia}
\subsection{Język python}
Do implementacji programu postanowiliśmy wykorzystać język python w wersji 3.10, ze względu na jego dynamiczność.
W tej wersji języka pojawił się również pattern-matching, który znacząco ułatwia pracę z ast.
\subsection{Parsowanie i tokenizowanie przy użyciu biblioteki sly}
Biblioteka sly, jest pythonową implementacją narzędzi lex i yacc, wykorzystywanych do tworzenia parserów i kompilatorów. Tworzenie leksera i parsera jest bardzo proste.


\subsubsection{Lexer}
Tokeny są tworzone przy pomocy wyrażeń regularnych.
\begin{lstlisting}
import sly
class Lex(sly.Lexer):
  ID = r"\w+"
  NUM = r"\d+"
\end{lstlisting}
Biblioteka udostępnia specjalny syntax do tworznia tokenów, które są już opisane jako inny token.
\begin{lstlisting}
  ID["and"] = AND
\end{lstlisting}
Pojedyńcze znaki mogą być ignorowane, poprzez ustawienie zmiennej `ignore'.
Dodatkowo ignorowane są też tokeny zaczynające się od `ignore'.
\begin{lstlisting}
  ignore = " \t"
  ignore_comm = r"\#.*"
\end{lstlisting}
Istnieje również możliwość tworzenia tokenów o typie równym ich wartości, tak długo jak składają się tylko z jedngo znaku.
\begin{lstlisting}
  literals = {"(", ")"}
\end{lstlisting}

\subsubsection{Parser}
W przypadku parsera każda reguła jest implementowana jako metoda, pod której argumentem mamy dostęp do tokenów i wartości zwróconych z innych metod.
\begin{lstlisting}
  import sly
  class Parser(sly.Parser):
  tokens = Lex.tokens
  
  @_("NUM")
  def expr(self, p):
  return p[0] # Lub p.ID
  
  @_("'(' expr AND expr ')'")
  def expr(self, p):
  return p.expr0 and p.expr1
\end{lstlisting}
Powyższy parser pozwala na opisanie wyrażeń logicznych typu: \lstinline{(1 and 0) and 4}.
\subsection{Środowisko nodejs do uruchomienia skompilowanego kodu}
Javascript jest językiem programowanie wykorzystywanym w przeglądarkach internetowych. W uruchomienia skompilowanego kodu używali będziemy jednak środowiska nodejs. Pozwoli to na szybkie i wygodne testowanie wygenerowanego kodu. Po zainstalowaniu środowiska i skompilowaniu pliku `index.uwu' otrzymamy plik `index.uwu.js', który możemy uruchomić w terminalu komendą `node index.uwu.js'.

\section{Implementacja}
\subsection{lexer}
Nasz lexer implementował będzie poniższy zbiór tokenów, warto zauważyć, że identyfikatory zaczynające się z dużej litery są identyfikatorami typów.
\lstinputlisting[firstline=75, lastline=94]{parser.py}
Nasz lexer zaimował się również będzie zwracaniem tokena dla znaku nowej linii
\lstinputlisting[firstline=99, lastline=102]{parser.py}
Lexer ignorował będzie tokeny komentarza, które zaczynają się od znaku `\#'. Oraz znaki tabulacji i spacji.
\newpage
\lstinputlisting[firstline=96, lastline=97]{parser.py}

Pozatym Nasz lexer zwracał będzie poniższy zbiór tokenów
\lstinputlisting[firstline=55, lastline=74]{parser.py}

\subsection{parser}
Implementacja parsera jest bardzo prosta, każda z metod zajmuje jedynie linijkę, gdzie zwracane jest odpowiedne wyrażenie drzewa ast.
\lstinputlisting[firstline=188, lastline=195]{parser.py}
W przypadku zbiorów wyrażeń, zwracane są listy.\newpage
\lstinputlisting[firstline=369, lastline=375]{parser.py}

\subsection{Drzewo decyzyjne}
Wykorzystywany algorytm jest odbrobinę inny od algorytmów opisanych w literaturze. Bierzemy pod uwagę następujące obserwacje. Część algorytmów w literaturze w celu uniknięcia wykładniczego wzrostu generowanego kodu, generuje zbędne testy \cite{Augustsson__1985}. Wykładniczy wzrostu generowanego jednak praktycznie nie wsystępuje w praktyce \cite{Scott__Ramsey__2000}. Najlepszą praktyką wieć wydało nam się by nigdy nie generować niepotrzebnych sprawdzeń, i spróbować uninkąć duplikacji kodu używając heurystyki tak jak \cite{Maranget__2008}. Literatura udowadnia, że w przypadku praktycznie napisanego kodu, różne algorytmy generują prawie identycny kod  \cite{Scott__Ramsey__2000, Maranget__2008}. Naszą metodą więc jest, aby: (1) zawsze skupiać się na przypasowaniu pierszego przypadku, by uniknąć niepotrzebnych testów i (2) zachłannie spróbować zminimalizować duplikację przy użyciu heurystyki.



Pattern matching zamieniany jest w drzewo decyzyjne przy pomocy rekurencyjnej funkcji.
Rekursja ma dwa podstawowe przypadki:
\begin{itemize}
  \item lista przypadków jest pusta, więc generujemy pustą gałąź końcową
  \item pierwszy przpadek, nie ma już więcej wzorów, więc zwracamy gałąź końcową z blokiem.
\end{itemize}
\lstinputlisting[firstline=72, lastline=80]{case_tree.py}
W innym przypadku wybieramy wzorzec przy pomocy heurystyki\newpage
\lstinputlisting[firstline=81, lastline=88]{case_tree.py}

Następnie tworzymy dwa podproblemy \lstinline{yes} i \lstinline{no} iterując przez wszsytki przypadki. Dla każdego z nich robimy jedną z trzech rzeczy:
\begin{itemize}
  \item Przypadek nie zawiera wzorca, więc dodajemy go do \lstinline{yes} i \lstinline{no}
        \lstinputlisting[firstline=90, lastline=95]{case_tree.py}
  \item Przpadek zawiera szukany wzorzec, więc dodajemy go do \lstinline{yes}
        \lstinputlisting[firstline=97, lastline=108]{case_tree.py}
  \item Przypadek zawiera wzorzec, więc dodajemy go do \lstinline{no}
        \lstinputlisting[firstline=110, lastline=111]{case_tree.py}
\end{itemize}

Rekursywnie generujemy kod dla \lstinline{yes} i \lstinline{no} i zwracamy gałąź decyzyjną
\lstinputlisting[firstline=116, lastline=118]{case_tree.py}
\subsection{inferencja typów}
\subsubsection{Context}
Implementację inferencji typów zaczynamy przez stworzenie zmiennego typu, który będzie reprezentowanty przez unikalny identyfikator.
\lstinputlisting[firstline=28, lastline=34]{algorithm_j.py}
Substytucje to pary zmiennych i typów
\lstinputlisting[firstline=25, lastline=25]{algorithm_j.py}
Jednym z argumentów algorytmu J jest środowiko, w naszej implementacji reprezentowane jest one przez Context, będący mapą identyfikatorów i schem
\lstinputlisting[firstline=26, lastline=26]{algorithm_j.py}
Gdzie Schema to zawieraja informację o typie, oraz listę występujących w nim zmiennych typów
\lstinputlisting[firstline=12, lastline=16]{algorithm_j.py}

\subsubsection{Aplikowanie substytucji}
Tworzymy funkcję, aplikującą substytucje na typie.

Przykład: dla subt=\{a:Num\} i ty=a->b funkcja zwraca Num -> b
\lstinputlisting[firstline=37, lastline=46]{algorithm_j.py}
Tworzymy funkcję, unifikującą dwa typy
\lstinputlisting[firstline=95, lastline=112]{algorithm_j.py}
wraz z funkcją generującą substytucje w przypadku unifikacji zmiennego typu
\lstinputlisting[firstline=115, lastline=124]{algorithm_j.py}
Teraz zaczynamy implementację algorytmu J.

\subsubsection{Literały}
W przypadku literałów typ, jest oczywisty.
\lstinputlisting[firstline=164, lastline=171]{algorithm_j.py}
\subsubsection{Identyfikatory}
W przypadku identyfikatorów, musimy zastąpić występujące w nich zmienne typy nowymi zmiennymi typami.
\lstinputlisting[firstline=172, lastline=173]{algorithm_j.py}
\subsubsection{Bloki wyrażeń}
W przypadku bloku wyrażeń inferujemy typ każdego z nich i zwracamy ostatni typ.
\lstinputlisting[firstline=185, lastline=192]{algorithm_j.py}

\subsubsection{Operatory binarne}
Implementacja operatorów binarnych są bardzo podobne.

Na przykład operator \lstinline{|} służącey do kontkatenacji dwóch list, oczekuje by wyrażenie z lewej i prawej strony były listami tego samego typu i zwraca nową listę tego właśnie typu.\newpage
\lstinputlisting[firstline=257, lastline=263]{algorithm_j.py}
\subsubsection{Wywowałnia funkcji}
W przypadku wywołania funkcji musimy zaimplementować currying.

Przykład: $J(fn)=Num\rightarrow Str\rightarrow Unit\rightarrow Arr<Str>\\ J(args)=[Num,Str]$
\begin{itemize}
  \item typy argumentów zbieramy w odwrotnej kolejności \lstinline{ty_args=[Str,Num]}
  \item Tworzymy nowy zmienny typ $ty=\alpha$
  \item Następnie przy użyciu funkcji \lstinline{reduce_args} aplikujemy \lstinline{ty_args} na typie $\alpha$ co tworzy typ $tmp=Num\rightarrow Str\rightarrow \alpha$
  \item Unifikujemy typy $tmp$ i $J(fn)$ co tworzy substytucję $\{\alpha: Unit\rightarrow Arr<Str>\}$
  \item I zwracamy $\alpha$
\end{itemize}
\lstinputlisting[firstline=323, lastline=335]{algorithm_j.py}

\subsubsection{Wyrażenia warunkowe}
W przypadku wyrażeń warunkowaych musimy, sprawdzić, czy warunek jest typu \lstinline{Bool} i czy, typy zwracane z każdej gałęzi są takie same, po czym zwracamy tą typ.
\lstinputlisting[firstline=361, lastline=373]{algorithm_j.py}

\subsubsection{Definicje funkcji}
W przypadku definicje funkcji musimy upenić się, że typy generyczne, są przypiswywane od contextu. Musimy sprawdzić, czy typ zwracany z funkcji jest przypisywalny do zadeklarowanego typu, oraz zapisać funkcję do contextu.
\lstinputlisting[firstline=336, lastline=360]{algorithm_j.py}

\subsection{Pattern mathing}
W trakcie pattern matching, musimy upewnić się, że konstruktory, posiadają odpowiednią ilość argumentów. Dodatkowo przeprowadzać będziemy sprawdzanie czy wyrażenie case jest wyczerpujące - przewiduje wszystkie przypadki. W tym celu funkcja przyjmuję argument \lstinline{o_alts} będący bazą, zmiennych i ich kontruktorów.

W przypadku pustej gałęzi końcowej, sprawdzamy, czy każda z list jest pusta i wyrzucamy błąd, jeżeli jest.
\lstinputlisting[firstline=508, lastline=520]{algorithm_j.py}

W przypadku gałęzi decyzyjnej inferujemy typ zmiennej i sprawdzanego konstruktora
\lstinputlisting[firstline=521, lastline=527]{algorithm_j.py}
Następne dwa kroki dopełniają nasze sprawdzenie czy wyrażenie jest wyczerpujące. Wpierw uzupełniamy naszą bazę konstruktorów przy użyciu funkcji \lstinline{alternatives} zwracjącej listę wszystkich konstruktorów dla danego typu.
Następnie usuwamy z bazy kontruktorów konstruktor do którego odnosi się nasza gałąź decyzjna.
\lstinputlisting[firstline=530, lastline=534]{algorithm_j.py}

Kolejna część inferuje typy zmiennych dla każdego z argumentów konstruktora.
Oraz, tworzy dla nich zmienne na kontekście.\newpage
\lstinputlisting[firstline=536, lastline=561]{algorithm_j.py}
Reszta funkcji to sprawdzenie, czy typy zwracane z bloków są takie same.
\lstinputlisting[firstline=564, lastline=578]{algorithm_j.py}


\subsection{Windowanie (Hoisting)}
Ze względu, na to, że deklaracje zmiennych w javascript \lstinline{const x = 1} w przeciwieństwie do naszego języka nie zwracają przypisywanej do indentyfikatora wartości, wyrażenia typy \lstinline{f(a = b = c)} muszą być przekonwertowane do
\begin{lstlisting}
  b=c
  a=b
  f(a)
\end{lstlisting}
W tym celu zaimplementowalismy funkcję, która przyjmuje wyrażenie i zwraca, listę wyrażeń do przeniesienia wyżej, oraz zmodyfikowane wyrażenie.

W przypadku bloków wyrażeń, listy z wyrażeniami do przeniesienia wyżej są łączone z listą wyrażeń danego bloku.
\lstinputlisting[firstline=46, lastline=56]{compile.py}
\lstinputlisting[firstline=13, lastline=21]{compile.py}

W przypadku deklaracji zmiennych i funkcji, dodajemy elementy do listy wyrażeń do przeniesienia wyżej.\newpage
\lstinputlisting[firstline=62, lastline=72]{compile.py}
W każdym innym przypadku lista wyrażeń do przeniesienia wyżej jest po prostu przepuszczana dalej.
\lstinputlisting[firstline=81, lastline=89]{compile.py}


\subsection{Kompilacja}
Kompilacja opiera się na jednej funkcji zamieniającej, ast na kod javascript w postaci stringa.

\subsection{Literały}
Jak widać, dla literałów jest to dość trywialne.
\lstinputlisting[firstline=145, lastline=152]{compile.py}

\subsubsection{Wyrażenia warunkowe}
Wyrażenia warunkowe, wymagają kompilacji, do postaci funkcji, gdyż w odróznieniu do javascript w naszym języku, zwracają one wartość.
\lstinputlisting[firstline=170, lastline=171]{compile.py}

\subsubsection{Warianty}
Warianty, przyjmują postać obiektów z dyskryminatorem
\lstinputlisting[firstline=187, lastline=190]{compile.py}

\subsubsection{Pattern matching}
W przypaku pustej gałęzi z blokiem kompilujemy blok.
W przypaku pustej gałęzi końcowej generujemy wyjątek.
\lstinputlisting[firstline=262, lastline=267]{compile.py}

W przypadku gałęzi decyzyjnej generowany jest blok if sprawdzający zawartość zmiennej. Można też uniknąć generowania bloku else, w przypadku, gdy wygenerowany byłby wyjątek.\newpage
\lstinputlisting[firstline=268, lastline=284]{compile.py}

\section{Opis działania}
Po uruchomieniu komendą \lstinline{python3 main.py **/*.uwu} programu, program skompiluje wszystkie pliki kończące się rozszerzeniem `.uwu' i wygeneruje dla każdego z nich odpowiedni plik z roszerzeniem `.js'.

\subsection{Co działa}
W programie udało się zaimplementować dużą część funkcjonalności potrzebnych do pisania faktycznych programów. Zabrakło jednak kilku początkowo planowanych funkcji: między innymi: rekursji, krotek, rekordów.
\subsection{Uwagi do działania}
\subsubsection{Obsługa błędów}
Błędy zwracane przez program nie są optymalne, nie zaimplementowaliśmy żadnej obsługi błędów w parserze. Błędy wynikające z inferencji typów, zwracają jedynie informacje o błędnych typach, nie o miejscu wystąpienia błędu. Wynika to z braku propagacji numerów linii i kolumn z parsera do drzewa ast.
\subsubsection{Pattern matching}
Pattern matching mógłby wyświetlać warning w momencie, gdy jedna ze ścieżek jest nieosiągalna.
\section{[Podsumowanie]}
\section{[spisy -- rysunków, tabel, listingów ipt.]}

\bibliographystyle{plainnat}

\bibliography{doc}


\end{document}


