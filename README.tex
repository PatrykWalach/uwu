% !TeX root = README.tex
\documentclass{article}
\usepackage[utf8]{inputenc}
\usepackage[T1]{fontenc}

\title{Tytuł}
\author{Patryk Wałach}
\date{January 2022}

\begin{document}

\maketitle

\section{[Wstęp]}

\section{Ogólne wprowadzenie}
\subsection{Parsowanie i tokenkizowanie}
\subsection{Co to typy?}
\section{Inferencja typów w teori}
\subsection{System typów Hindley–Milner}
\subsection{Algorytm W}
\section{Rescript/ReasonML jako języki realizujące podobne zadania}
\section{Założenia i priorytety opracowanej aplikacji}
Ta strona została opracowana w celu przedstawienia założeń i priorytetów opracowanej aplikacji..
\subsection{Opis formaly składni języka}
%  notacja wirta
\section{Narzędzia}
\subsection{Język python}
\subsection{Parsowanie i tokenizowanie przy użyciu biblioteki sly}
\subsection{Środowisko nodejs do uruchomienia skompilowanego kodu}
\section{Implementacja}
\subsection{lexer}
\subsection{parser}
\subsection{inferencja typów}
\subsection{kompilacja}
\section{Opis działania}
\subsection{Co działa}
\subsection{Uwagi co do obsługi błędów}
\section{[Podsumowanie]}
\section{[spisy -- rysunków, tabel, listingów ipt.]}


% \cite{texbook}
% \bibitem{texbook}

\end{document}


